\begin{spell}{Dream of the Blue Veil}{TCE}{7th-level conjuration}
{
	\spTime{10 minutes}
	\spRange{20 feet}
	\spComponents{V, S, M (a magic item or a willing creature from the destination world)}
	\spDuration{6 hours}
}
You and up to eight willing creatures within range fall unconscious for the spell's duration and experience visions of another world on the Material Plane, such as Oerth, Toril, Krynn, or Eberron. If the spell reaches its full duration, the visions conclude with each of you encountering and pulling back a mysterious blue curtain. The spell then ends with you mentally and physically transported to the world that was in the visions.
To cast this spell, you must have a magic item that originated on the world you wish to reach, and you must be aware of the world’s existence, even if you don’t know the world’s name. Your destination in the other world is a safe location within 1 mile of where the magic item was created. Alternatively, you can cast the spell if one of the affected creatures was born on the other world, which causes your destination to be a safe location within 1 mile of where that creature was born.
The spell ends early on a creature if that creature takes any damage, and the creature isn’t transported. If you take any damage, the spell ends for you and all other creatures, with none of you being transported.
Spell Lists. Bard (Optional), Sorcerer (Optional), Warlock (Optional), Wizard (Optional)
\end{spell}

\begin{spell}{Etherealness}{PHB}{7th-level transmutation}
{
	\spTime{1 action}
	\spRange{Self}
	\spComponents{V, S}
	\spDuration{Up to 8 hours}
}
You step into the border regions of the Ethereal Plane, in the area where it overlaps with your current plane. You remain in the Border Ethereal for the duration or until you use your action to dismiss the spell. During this time, you can move in any direction. If you move up or down, every foot of movement costs an extra foot. You can see and hear the plane you originated from, but everything there looks gray, and you can’t see anything more than 60 feet away.
While on the Ethereal Plane, you can only affect and be affected by other creatures on that plane. Creatures that aren’t on the Ethereal Plane can’t perceive you and can’t interact with you, unless a special ability or magic has given them the ability to do so.
You ignore all objects and effects that aren’t on the Ethereal Plane, allowing you to move through objects you perceive on the plane you originated from. When the spell ends, you immediately return to the plane you originated from in the spot you currently occupy. If you occupy the same spot as a solid object or creature when this happens, you are immediately shunted to the nearest unoccupied space that you can occupy and take force damage equal to twice the number of feet you are moved.
This spell has no effect if you cast it while you are on the Ethereal Plane or a plane that doesn’t border it, such as one of the Outer Planes.
At Higher Levels. When you cast this spell using a spell slot of 8th level or higher, you can target up to three willing creatures (including you) for each slot level above 7th. The creatures must be within 10 feet of you when you cast the spell.
Spell Lists. Bard, Cleric, Sorcerer, Warlock, Wizard

\note{At Higher Levels.} When you cast this spell using a spell slot of 8th level or higher, you can target up to three willing creatures (including you) for each slot level above 7th. The creatures must be within 10 feet of you when you cast the spell.
\end{spell}

\begin{spell}{Forcecage}{PHB}{7th-level evocation}
{
	\spTime{1 action}
	\spRange{100 feet}
	\spComponents{V, S, M (ruby dust worth 1,500 gp)}
	\spDuration{1 hour}
}
An immobile, invisible, cube-shaped prison composed of magical force springs into existence around an area you choose within range. The prison can be a cage or a solid box as you choose.
A prison in the shape of a cage can be up to 20 feet on a side and is made from 1/2-inch diameter bars spaced 1/2 inch apart.
A prison in the shape of a box can be up to 10 feet on a side, creating a solid barrier that prevents any matter from passing through it and blocking any spells cast into or out of the area.
When you cast the spell, any creature that is completely inside the cage's area is trapped. Creatures only partially within the area, or those too large to fit inside the area, are pushed away from the center of the area until they are completely outside the area.
A creature inside the cage can't leave it by nonmagical means. If the creature tries to use teleportation or interplanar travel to leave the cage, it must first make a Charisma saving throw. On a success, the creature can use that magic to exit the cage. On a failure, the creature can't exit the cage and wastes the use of the spell or effect. The cage also extends into the Ethereal Plane, blocking ethereal travel.
This spell can't be dispelled by dispel magic.
Spell Lists. Bard, Warlock, Wizard
\end{spell}

\begin{spell}{Mirage Arcane}{PHB}{7th-level illusion}
{
	\spTime{10 minutes}
	\spRange{Sight}
	\spComponents{V, S}
	\spDuration{10 days}
}
You make terrain in an area up to 1 mile square look, sound, smell, and even feel like some other sort of terrain. The terrain’s general shape remains the same, however. Open fields or a road could be made to resemble a swamp, hill, crevasse, or some other difficult or impassable terrain. A pond can be made to seem like a grassy meadow, a precipice like a gentle slope, or a rock-strewn gully like a wide and smooth road.
Similarly, you can alter the appearance of structures, or add them where none are present. The spell doesn’t disguise, conceal, or add creatures.
The illusion includes audible, visual, tactile, and olfactory elements, so it can turn clear ground into difficult terrain (or vice versa) or otherwise impede movement through the area. Any piece of the illusory terrain (such as a rock or stick) that is removed from the spell’s area disappears immediately.
Creatures with truesight can see through the illusion to the terrain’s true form; however, all other elements of the illusion remain, so while the creature is aware of the illusion’s presence, the creature can still physically interact with the illusion.
Spell Lists. Bard, Druid, Wizard
\end{spell}

\begin{spell}{Mordenkainen's Magnificent Mansion}{PHB}{7th-level conjuration}
{
	\spTime{1 minute}
	\spRange{300 feet}
	\spComponents{V, S, M (a miniature portal carved from ivory, a small piece of polished marble, and a tiny silver spoon, each item worth at least 5 gp)}
	\spDuration{24 hours}
}
You conjure an extradimensional dwelling in range that lasts for the duration. You choose where its one entrance is located. The entrance shimmers faintly and is 5 feet wide and 10 feet tall. You and any creature you designate when you cast the spell can enter the extradimensional dwelling as long as the portal remains open. You can open or close the portal if you are within 30 feet of it. While closed, the portal is invisible.
Beyond the portal is a magnificent foyer with numerous chambers beyond. The atmosphere is clean, fresh, and warm.
You can create any floor plan you like, but the space can’t exceed 50 cubes, each cube being 10 feet on each side. The place is furnished and decorated as you choose. It contains sufficient food to serve a nine-course banquet for up to 100 people. A staff of 100 near-transparent servants attends all who enter. You decide the visual appearance of these servants and their attire. They are completely obedient to your orders. Each servant can perform any task a normal human servant could perform, but they can’t attack or take any action that would directly harm another creature. Thus the servants can fetch things, clean, mend, fold clothes, light fires, serve food, pour wine, and so on. The servants can go anywhere in the mansion but can’t leave it. Furnishings and other objects created by this spell dissipate into smoke if removed from the mansion. When the spell ends, any creatures or objects inside the extradimensional space are expelled into the open spaces nearest to the entrance.
Spell Lists. Bard, Wizard
\end{spell}

\begin{spell}{Mordenkainen's Sword}{PHB}{7th-level evocation}
{
	\spTime{1 action}
	\spRange{60 feet}
	\spComponents{V, S, M (a miniature platinum sword with a grip and pommel of copper and zinc, worth 250 gp)}
	\spDuration{Concentration, up to 1 minute}
}
You create a sword-shaped plane of force that hovers within range. It lasts for the duration.
When the sword appears, you make a melee spell attack against a target of your choice within 5 feet of the sword. On a hit. the target takes 3d10 force damage. Until the spell ends, you can use a bonus action on each of your turns to move the sword up to 20 feet to a spot you can see and repeat this attack against the same target or a different one.
Spell Lists. Bard, Wizard
\end{spell}

\begin{spell}{Prismatic Spray}{PHB}{7th-level evocation}
{
	\spTime{1 action}
	\spRange{Self (60-foot cone)}
	\spComponents{V, S}
	\spDuration{Instantaneous}
}
Eight multicolored rays of light flash from your hand. Each ray is a different color and has a different power and purpose. Each creature in a 60-foot cone must make a Dexterity saving throw. For each target, roll a d8 to determine which color ray affects it.
Spell Lists. Bard, Sorcerer, Wizard
\end{spell}

\begin{spell}{Project Image}{PHB}{7th-level illusion}
{
	\spTime{1 action}
	\spRange{500 miles}
	\spComponents{V, S, M (a small replica of you made from materials worth at least 5 gp)}
	\spDuration{Concentration, up to 1 day}
}
You create an illusory copy of yourself that lasts for the duration. The copy can appear at any location within range that you have seen before, regardless of intervening obstacles. The illusion looks and sounds like you but is intangible. If the illusion takes any damage, it disappears, and the spell ends.
You can use your action to move this illusion up to twice your speed, and make it gesture, speak, and behave in whatever way you choose. It mimics your mannerisms perfectly.
You can see through its eyes and hear through its ears as if you were in its space. On your turn as a bonus action, you can switch from using its senses to using your own, or back again. While you are using its senses, you are blinded and deafened in regard to your own surroundings.
Physical interaction with the image reveals it to be an illusion, because things can pass through it. A creature that uses its action to examine the image can determine that it is an illusion with a successful Intelligence (Investigation) check against your spell save DC. If a creature discerns the illusion for what it is, the creature can see through the image, and any noise it makes sounds hollow to the creature.
Spell Lists. Bard, Wizard
\end{spell}

\begin{spell}{Regenerate}{PHB}{7th-level transmutation}
{
	\spTime{1 minute}
	\spRange{Touch}
	\spComponents{V, S, M (a prayer wheel and holy water)}
	\spDuration{1 hour}
}
You touch a creature and stimulate its natural healing ability. The target regains 4d8 + 15 hit points. For the duration of the spell, the target regains 1 hit point at the start of each of its turns (10 hit points each minute).
The target’s severed body members (fingers, legs, tails, and so on), if any, are restored after 2 minutes. If you have the severed part and hold it to the stump, the spell instantaneously causes the limb to knit to the stump.
Spell Lists. Bard, Cleric, Druid
\end{spell}

\begin{spell}{Resurrection}{PHB}{7th-level necromancy}
{
	\spTime{1 hour}
	\spRange{Touch}
	\spComponents{V, S, M (a diamond worth at least 1,000 gp, which the spell consumes)}
	\spDuration{Instantaneous}
}
You touch a dead creature that has been dead for no more than a century, that didn’t die of old age, and that isn’t undead. If its soul is free and willing, the target returns to life with all its hit points.
This spell neutralizes any poisons and cures normal diseases afflicting the creature when it died. It doesn’t, however, remove magical diseases, curses, and the like; if such affects aren’t removed prior to casting the spell, they afflict the target on its return to life.
This spell closes all mortal wounds and restores any missing body parts.
Coming back from the dead is an ordeal. The target takes a -4 penalty to all attack rolls, saving throws, and ability checks. Every time the target finishes a long rest, the penalty is reduced by 1 until it disappears.
Casting this spell to restore life to a creature that has been dead for one year or longer taxes you greatly. Until you finish a long rest, you can’t cast spells again, and you have disadvantage on all attack rolls, ability checks, and saving throws.
Spell Lists. Bard, Cleric
\end{spell}

\begin{spell}{Symbol}{PHB}{7th-level abjuration}
{
	\spTime{1 minute}
	\spRange{Touch}
	\spComponents{V, S, M (mercury, phosphorus, and powdered diamond and opal with a total value of at least 1,000 gp, which the spell consumes)}
	\spDuration{Until dispelled or triggered}
}
When you cast this spell, you inscribe a harmful glyph either on a surface (such as a section of floor, a wall, or a table) or within an object that can be closed to conceal the glyph (such as a book, a scroll, or a treasure chest). If you choose a surface, the glyph can cover an area of the surface no larger than 10 feet in diameter. If you choose an object, that object must remain in its place; if the object is moved more than 10 feet from where you cast this spell, the glyph is broken, and the spell ends without being triggered.
The glyph is nearly invisible, requiring an Intelligence (Investigation) check against your spell save DC to find it.
You decide what triggers the glyph when you cast the spell. For glyphs inscribed on a surface, the most typical triggers include touching or stepping on the glyph, removing another object covering it, approaching within a certain distance of it, or manipulating the object that holds it. For glyphs inscribed within an object, the most common triggers are opening the object, approaching within a certain distance of it, or seeing or reading the glyph.
You can further refine the trigger so the spell is activated only under certain circumstances or according to a creature's physical characteristics (such as height or weight), or physical kind (for example, the ward could be set to affect hags or shapechangers). You can also specify creatures that don't trigger the glyph, such as those who say a certain password.
When you inscribe the glyph, choose one of the options below for its effect. Once triggered, the glyph glows, filling a 60-foot-radius sphere with dim light for 10 minutes, after which time the spell ends. Each creature in the sphere when the glyph activates is targeted by its effect, as is a creature that enters the sphere for the first time on a turn or ends its turn there.
Spell Lists. Bard, Cleric, Druid (Optional), Wizard
\end{spell}

\begin{spell}{Teleport}{PHB}{7th-level conjuration}
{
	\spTime{1 action}
	\spRange{10 feet}
	\spComponents{V}
	\spDuration{Instantaneous}
}
This spell instantly transports you and up to eight willing creatures of your choice that you can see within range, or a single object that you can see within range, to a destination you select. If you target an object, it must be able to fit entirely inside a 10-foot cube, and it can’t be held or carried by an unwilling creature.
The destination you choose must be known to you, and it must be on the same plane of existence as you. Your familiarity with the destination determines whether you arrive there successfully. The DM rolls d100 and consults the table.
"Permanent circle" means a permanent Teleportation Circle whose sigil sequence you know.
"Associated object" means that you possess an object taken from the desired destination within the last six months, such as a book from a wizard's library, bed linen from a royal suite, or a chunk of marble from a lich's secret tomb.
"Very familiar" is a place you have been very often, a place you have carefully studied, or a place you can see when you cast the spell.
"Seen casually" is someplace you have seen more than once but with which you aren't very familiar.
"Viewed once" is a place you have seen once, possibly using magic.
"Description" is a place whose location and appearance you know through someone else's description, perhaps from a map.
"False destination" is a place that doesn't exist. Perhaps you tried to scry an enemy's sanctum but instead viewed an illusion, or you are attempting to teleport to a familiar location that no longer exists.
Spell Lists. Bard, Sorcerer, Wizard
\end{spell}

