\begin{spell}{Charm Monster}{XGE}{4th-level enchantment}
{
	\spTime{1 action}
	\spRange{30 feet}
	\spComponents{V, S}
	\spDuration{1 hour}
}
You attempt to charm a creature you can see within range. It must make a Wisdom saving throw, and it does so with advantage if you or your companions are fighting it. If it fails the saving throw, it is charmed by you until the spell ends or until you or your companions do anything harmful to it. The charmed creature is friendly to you. When the spell ends, the creature knows it was charmed by you.
At Higher Levels. When you cast this spell using a spell slot of 5th level or higher, you can target one additional creature for each slot level above 4th. The creatures must be within 30 feet of each other when you target them.
Spell Lists. Bard, Druid, Sorcerer, Warlock, Wizard

\note{At Higher Levels.} When you cast this spell using a spell slot of 5th level or higher, you can target one additional creature for each slot level above 4th. The creatures must be within 30 feet of each other when you target them.
\end{spell}

\begin{spell}{Compulsion}{PHB}{4th-level enchantment}
{
	\spTime{1 action}
	\spRange{30 feet}
	\spComponents{V, S}
	\spDuration{Concentration, up to 1 minute}
}
Creatures of your choice that you can see within range and that can hear you must make a Wisdom saving throw. A target automatically succeeds on this saving throw if it can’t be charmed. On a failed save, a target is affected by this spell. Until the spell ends, you can use a bonus action on each of your turns to designate a direction that is horizontal to you. Each affected target must use as much of its movement as possible to move in that direction on its next turn. It can take its action before it moves. After moving in this way, it can make another Wisdom saving throw to try to end the effect.
A target isn’t compelled to move into an obviously deadly hazard, such as a fire pit, but it will provoke opportunity attacks to move in the designated direction.
Spell Lists. Bard
\end{spell}

\begin{spell}{Confusion}{PHB}{4th-level enchantment}
{
	\spTime{1 action}
	\spRange{90 feet}
	\spComponents{V, S, M (three nut shells)}
	\spDuration{Concentration, up to 1 minute}
}
This spell assaults and twists creatures’ minds, spawning delusions and provoking uncontrolled actions. Each creature in a 10-foot-radius sphere centered on a point you choose within range must succeed on a Wisdom saving throw when you cast this spell or be affected by it.
An affected target can’t take reactions and must roll a d10 at the start of each of its turns to determine its behavior for that turn.
At the end of its turns, an affected target can make a Wisdom saving throw. If it succeeds, this effect ends for that target.
At Higher Levels. When you cast this spell using a spell slot of 5th level or higher, the radius of the sphere increases by 5 feet for each slot level above 4th.
Spell Lists. Bard, Druid, Sorcerer, Wizard

\note{At Higher Levels.} When you cast this spell using a spell slot of 5th level or higher, the radius of the sphere increases by 5 feet for each slot level above 4th.
\end{spell}

\begin{spell}{Dimension Door}{PHB}{4th-level conjuration}
{
	\spTime{1 action}
	\spRange{500 feet}
	\spComponents{V}
	\spDuration{Instantaneous}
}
You teleport yourself from your current location to any other spot within range. You arrive at exactly the spot desired. It can be a place you can see, one you can visualize, or one you can describe by stating distance and direction, such as "200 feet straight downward" or "upward to the northwest at a 45-degree angle, 300 feet".
You can bring along objects as long as their weight doesn’t exceed what you can carry. You can also bring one willing creature of your size or smaller who is carrying gear up to its carrying capacity. The creature must be within 5 feet of you when you cast this spell.
If you would arrive in a place already occupied by an object or a creature, you and any creature traveling with you each take 4d6 force damage, and the spell fails to teleport you.
Spell Lists. Bard, Sorcerer, Warlock, Wizard
\end{spell}

\begin{spell}{Ego Whip (UA)}{UA-66}{4th-level enchantment}
{
	\spTime{1 action}
	\spRange{30 feet}
	\spComponents{V}
	\spDuration{Concentration, up to 1 minute}
}
You lash the mind of one creature you can see within range, filling it with despair. The target must succeed on an Intelligence saving throw or suffer disadvantage on attack rolls, ability checks, and saving throws, and it can’t cast spells. At the end of each of its turns, the target can make another Intelligence saving throw. On a success, the spell ends on the target.
Spell Lists. Bard, Sorcerer, Warlock, Wizard
\end{spell}

\begin{spell}{Freedom of Movement}{PHB}{4th-level abjuration}
{
	\spTime{1 action}
	\spRange{Touch}
	\spComponents{V, S, M (a leather strap, bound around the arm or a similar appendage)}
	\spDuration{1 hour}
}
You touch a willing creature. For the duration, the target’s movement is unaffected by difficult terrain, and spells and other magical effects can neither reduce the target’s speed nor cause the target to be paralyzed or restrained.
The target can also spend 5 feet of movement to automatically escape from nonmagical restraints, such as manacles or a creature that has it grappled. Finally, being underwater imposes no penalties on the target’s movement or attacks.
Spell Lists. Artificer, Bard, Cleric, Druid, Ranger
\end{spell}

\begin{spell}{Greater Invisibility}{PHB}{4th-level illusion}
{
	\spTime{1 action}
	\spRange{Touch}
	\spComponents{V, S}
	\spDuration{Concentration, up to 1 minute}
}
You or a creature you touch becomes invisible until the spell ends. Anything the target is wearing or carrying is invisible as long as it is on the target’s person.
Spell Lists. Bard, Sorcerer, Wizard
\end{spell}

\begin{spell}{Hallucinatory Terrain}{PHB}{4th-level illusion}
{
	\spTime{10 minutes}
	\spRange{300 feet}
	\spComponents{V, S, M (a stone, a twig, and a bit of green plant)}
	\spDuration{24 hours}
}
You make natural terrain in a 150-foot cube in range look, sound, and smell like some other sort of natural terrain. Thus, open fields or a road can be made to resemble a swamp, hill, crevasse, or some other difficult or impassable terrain. A pond can be made to seem like a grassy meadow, a precipice like a gentle slope, or a rock-strewn gully like a wide and smooth road. Manufactured structures, equipment, and creatures within the area aren’t changed in appearance.
The tactile characteristics of the terrain are unchanged, so creatures entering the area are likely to see through the illusion. If the difference isn’t obvious by touch, a creature carefully examining the illusion can attempt an Intelligence (Investigation) check against your spell save DC to disbelieve it. A creature who discerns the illusion for what it is, sees it as a vague image superimposed on the terrain.
Spell Lists. Bard, Druid, Warlock, Wizard
\end{spell}

\begin{spell}{Locate Creature}{PHB}{4th-level divination}
{
	\spTime{1 action}
	\spRange{Self}
	\spComponents{V, S, M (a bit of fur from a bloodhound)}
	\spDuration{Concentration, up to 1 hour}
}
Describe or name a creature that is familiar to you. You sense the direction to the creature’s location, as long as that creature is within 1,000 feet of you. If the creature is moving, you know the direction of its movement.
The spell can locate a specific creature known to you, or the nearest creature of a specific kind (such as a human or a unicorn), so long as you have seen such a creature up close – within 30 feet – at least once. If the creature you described or named is in a different form, such as being under the effects of a polymorph spell, this spell doesn’t locate the creature.
This spell can’t locate a creature if running water at least 10 feet wide blocks a direct path between you and the creature.
Spell Lists. Bard, Cleric, Druid, Paladin, Ranger, Wizard
\end{spell}

\begin{spell}{Phantasmal Killer}{PHB}{4th-level illusion}
{
	\spTime{1 action}
	\spRange{120 feet}
	\spComponents{V, S}
	\spDuration{Concentration, up to 1 minute}
}
You tap into the nightmares of a creature you can see within range and create an illusory manifestation of its deepest fears, visible only to that creature.
The target must make a Wisdom saving throw. On a failed save, the target becomes frightened for the duration. At the end of each of the target’s turns before the spell ends, the target must succeed on a Wisdom saving throw or take 4d10 psychic damage. On a successful save, the spell ends.
At Higher Levels. When you cast this spell using a spell slot of 5th level or higher, the damage increases by 1d10 for each slot level above 4th.
Spell Lists. Bard (Optional), Wizard

\note{At Higher Levels.} When you cast this spell using a spell slot of 5th level or higher, the damage increases by 1d10 for each slot level above 4th.
\end{spell}

\begin{spell}{Polymorph}{PHB}{4th-level transmutation}
{
	\spTime{1 action}
	\spRange{60 feet}
	\spComponents{V, S, M (a caterpillar cocoon)}
	\spDuration{Concentration, up to 1 hour}
}
This spell transforms a creature that you can see within range into a new form. An unwilling creature must make a Wisdom saving throw to avoid the effect. A shapechanger automatically succeeds on this saving throw.
The transformation lasts for the duration, or until the target drops to 0 hit points or dies. The new form can be any beast whose challenge rating is equal to or less than the target’s (or the target’s level, if it doesn’t have a challenge rating). The target’s game statistics, including mental ability scores, are replaced by the statistics of the chosen beast. It retains its alignment and personality.
The target assumes the hit points of its new form. When it reverts to its normal form, the creature returns to the number of hit points it had before it transformed. If it reverts as a result of dropping to 0 hit points, any excess damage carries over to its normal form. As long as the excess damage doesn’t reduce the creature’s normal form to 0 hit points, it isn’t knocked unconscious.
The creature is limited in the actions it can perform by the nature of its new form, and it can’t speak, cast spells, or take any other action that requires hands or speech.
The target’s gear melds into the new form. The creature can’t activate, use, wield, or otherwise benefit from any of its equipment. This spell can’t affect a target that has 0 hit points.
Spell Lists. Bard, Druid, Sorcerer, Wizard
\end{spell}

\begin{spell}{Raulothim's Psychic Lance}{FTD}{4th-level Enchantment}
{
	\spTime{1 action}
	\spRange{120 feet}
	\spComponents{V}
	\spDuration{Instantaneous}
}
You unleash a shimmering lance of psychic power from your forehead at a creature that you can see within range. Alternatively, you can utter a creature’s name. If the named target is within range, it becomes the spell’s target even if you can’t see it. If the named target isn’t within range, the lance dissipates without effect.
The target must make an Intelligence saving throw. On a failed save, the target takes 7d6 psychic damage and is incapacitated until the start of your next turn. On a successful save, the creature takes half as much damage and isn’t incapacitated.
At Higher Levels. When you cast this spell using a spell slot of 5th level or higher, the damage increases by 1d6 for each slot level above 4th.
Spell Lists. Bard, Sorcerer, Warlock, Wizard

\note{At Higher Levels.} When you cast this spell using a spell slot of 5th level or higher, the damage increases by 1d6 for each slot level above 4th.
\end{spell}

\begin{spell}{Raulothim's Psychic Lance (UA)}{UA-78}{4th-level enchantment}
{
	\spTime{1 action}
	\spRange{120 feet}
	\spComponents{V}
	\spDuration{Instantaneous}
}
You unleash a shimmering lance of psychic power from your forehead at a creature that you can see within range. Alternatively, you can utter the creature’s name. If the named target is within range, it gains no benefit from cover or invisibility as the lance homes in on it. If the named target isn’t within range, the lance dissipates, and the spell slot is not expended.
The target must succeed on an Intelligence saving throw or take 10d6 psychic damage and be incapacitated until the start of your next turn.
At Higher Levels. When you cast this spell using a spell slot of 5th level or higher, the damage increases by 1d6 for each slot level above 4th.
Spell Lists. Bard, Sorcerer, Warlock, Wizard

\note{At Higher Levels.} When you cast this spell using a spell slot of 5th level or higher, the damage increases by 1d6 for each slot level above 4th.
\end{spell}

